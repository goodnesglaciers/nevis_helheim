\documentclass[a4paper,11pt,fleqn]{article}
\title{}
\author{I J Hewitt}
\date{\today}
\addtolength{\voffset}{-2cm}
\addtolength{\textheight}{4.2cm}
\addtolength{\hoffset}{-2cm}
\addtolength{\textwidth}{4cm}
\usepackage{graphicx}
\usepackage{epstopdf}
\usepackage{epsfig}
\usepackage{color}
\usepackage{amsmath}
\usepackage{amssymb}
\usepackage{verbatim}
\usepackage{fancyhdr}
\usepackage{natbib}
\usepackage{parskip}
%\usepackage{hyperref}
\usepackage{url}

\setlength{\headheight}{10pt}
\pagestyle{fancy}
\renewcommand{\headrulewidth}{1pt}
\renewcommand{\footrulewidth}{0pt}
\fancyhf{}
\cfoot{\thepage}

\fancypagestyle{first}{
\setlength{\headheight}{20pt}
\renewcommand{\headrulewidth}{1pt}
\renewcommand{\footrulewidth}{0pt}
\fancyhf{}
\rhead{{\small I J Hewitt, August 2014}}
%\rfoot{{Corrections: hewitt@maths.ox.ac.uk}}
\cfoot{\thepage}
}

\newcommand{\pvint}{\makebox[0pt][l]{$\:-$}\int}
\newcommand{\un}[1]{\ \textrm{#1}}
\newcommand{\ve}[1]{\bm{#1}}
\newcommand{\be}{\begin{equation} }
\newcommand{\ee}{\end{equation}}
\newcommand{\bmath}{\begin{displaymath} }
\newcommand{\emath}{\end{displaymath} }
\newcommand{\lab}[1]{\label{#1}}
\newcommand{\eps}{\varepsilon}
\newcommand{\pd}[2]{\frac{\partial #1}{\partial #2}}
\newcommand{\bm}[1]{\mbox{\boldmath$ #1 $}}
\newcommand{\bo}[1]{\mbox{\bf #1}}
\newcommand{\quadtext}[1]{\quad\textrm{#1}\quad}
\newcommand{\ian}[1]{{\bf [#1]}}
\newcommand{\half}{\mbox{$\frac{1}{2}$}}
\newcommand{\dd}{\textrm{d}}



\begin{document}

\thispagestyle{first}

{\setlength{\parskip}{-5pt}
{\Large \bf \flushleft{ {\verb|nevis|} }: Documentation
} \setlength{\parskip}{5pt}

{This model is mostly as described in Hewitt (2013).\footnote{Hewitt, I.J. 2013 Seasonal Changes in Ice Sheet Motion due to Meltwater Lubrication {\it Earth and Planetary Science Letters}, {\bf 371--372}, 16-25.}  Continuum variables on a two-dimensional domain (hydraulic potential $\phi$ and water sheet thickness $h$) are discretised on a rectangular mesh of nodes.  These nodes also form the basis of a discrete network of conduits (cross-sectional area $S$), connecting the nodes across each of the 8 compass points.  Water is exchanged between englacial storage, water sheet and conduits such that the hydraulic potential is continuous.  

\subsection*{Equations}

Variables and parameters are summarized in Table \ref{tab1}. 

Atmospheric and overburden potentials (corresponding to $p_w=0$ and $p_w = p_i \equiv \rho_i g (Z_s-Z_b)$ respectively) are defined by
\be
\phi_a(x,y) = \rho_w g Z_b, \qquad
\phi_0(x,y) = \rho_i g Z_s + (\rho_w-\rho_i) g Z_b.
\ee
Ice pressure, water pressure and effective pressure are related to hydraulic potential $\phi(x,y,t)$ through
\be
p_i(x,y) = \phi_0 - \phi_a, \qquad 
p_w(x,y,t) = \phi - \phi_a, \qquad 
N(x,y,t) = \phi_0 - \phi.
\ee


The distributed sheet is separated into a cavity sheet, with thickness $h_{cav}(x,y,t)$, and an elastic sheet, with thickness $h_{el}(x,y,t)$. By default these are added together to obtain the overall sheet thickness $h(x,y,t) = h_{cav} + h_{el}$, though they can alternatively be treated separately, with different discharge in each. The cavity sheet evolves according to
\be
\pd{h_{cav}}{t} = \frac{\rho_w}{\rho_i} m + U_b(h_r - h_{cav})_+/\ell_r - \hat{A} h_{cav} |N|^{n-1} N.
\ee
The elastic sheet has thickness related directly to the water pressure
\be
h_{el} = h_c \left(\frac{p_w}{p_i}\right)^{\gamma} + h_{\eps 2} \left[ -N_- + \half N_{\eps 2} \left(1-N_+/N_{\eps 2}\right)_+^2 \right],
\ee
where $N_- = \min(N,0)$ and $N_+ = \max(N,0)$.  The second term here is designed to increase rapidly when $N$ is negative, as a crude representation of hydraulic jacking.  Discharge $\bo{q}(x,y,t)$ is given by
\be
{\bo q} = -K_s h^{\alpha_s} |\nabla \phi|^{\beta_s-1} \nabla \phi.
\ee
Basal melting in the sheet $m(x,y)$ is either prescribed, or should be coupled to ice sliding speed and basal shear stress according to (any conduction into the ice is absorbed into $G$ here)
\be
m =  \frac{G + {\bm \tau}_b\cdot{\bf u}_b}{\rho_w L}.
\ee


Each conduit has cross-sectional area $S(s,t)$ ($s$ is distance along the conduit) evolving according to 
\be
\pd{S}{t} = \frac{\rho_w}{\rho_i} M +  U_b h_{rc} (1 - S/S_{rc})_+ - \tilde{A} S |N|^{n-1}N .
\ee
Discharge $Q(s,t)$ is given by
\be
Q = -K_c S^{\alpha_c} \left|\pd{\phi}{s}\right|^{\beta_c-1/2} \pd{\phi}{s}.
\ee
Melting $M(s,t)$ is given by
\be
M =  \frac{1-\rho_w c \beta}{\rho_w L} \left|Q \pd{\phi}{s}\right| - \frac{\rho_w g c \beta}{L} Q \pd{Z_b}{s} + \lambda \frac{|\bo{q} \cdot \nabla \phi |}{\rho_w L}.
\ee

Mass conservation is expressed as
\be
\pd{h}{t} + \nabla \cdot \bo{q} + \left[ \pd{S}{t} + \pd{Q}{s} \right] \delta(\bo{x}_c) + \pd{\Sigma}{t} = m + M \delta(\bo{x}_c) + E,
\ee
where englacial storage $\Sigma(x,y,t)$ is a function of water pressure 
\be
\Sigma = \sigma\frac{p_w}{\rho_w g} + A_m \frac{p_w}{\rho_w g}\delta(\bo{x}_m),
\ee
and the delta functions apply along the (linear) positions of the conduits $\bo{x}_c(s)$ and the (point) positions of the moulins $\bo{x}_m$.  The source term $E(x,y,t)$ can include both distributed and moulin point sources.

Boundary conditions are required on either the pressure or the discharge at all boundaries of the domain.  Initial conditions are required for $\phi$, $h$ and $S$.

\begin{table}[!h]
\centering
{\small 
\begin{tabular}{l l}
\hline
Primary variables [\verb|vv|]&
\begin{tabular}[t]{l l l}
$\phi(x,y,t)$  	&\verb|phi|&	 Hydraulic potential\\
$h_s(x,y,t)$ 	&\verb|hs| & 	 Cavity sheet depth \\
$S(s,t)$ 		&\verb|Sx|,\verb|Sy|,\verb|Ss|,\verb|Sr| & 	 Conduit cross-sectional area \\
\end{tabular}\\
\hline
& \\[-10pt]
\hline
Derived variables [\verb|vv|]&
\begin{tabular}[t]{l l l}
$h_e(x,y,t)$ 	&\verb|he| &  	Elastic sheet depth \\
$N(x,y,t)$ 		&\verb|N| &  	Effective pressure in sheet \\
$Q(s,t)$ 		&\verb|Qx|,\verb|Qy|,\verb|Qs|,\verb|Qr| &  	Discharge in channel \\
$\bo{q}(x,y,t)$ 	&\verb|qx|,\verb|qy| & 	 Discharge in sheet \\
\end{tabular} \\
\hline
& \\[-10pt]
\hline
Prescribed fields [\verb|aa|]&
\begin{tabular}[t]{l l l}
$Z_b(x,y)$ 		&\verb|b|&  	Bed elevation \\ 
$Z_s(x,y)$ 		&\verb|s|& 	Surface elevation \\ 
$\phi_a(x,y)$ 		&\verb|phi_a|&  Atmospheric potential at bed elevation \\ 
$\phi_0(x,y)$ 		&\verb|phi_0|& 	Overburden potential \\ 
$m(x,y)$ 			&\verb|m| &  	Basal source\\
$E(x,y,t)$ 			&\verb|E| & 	Englacial source\\
$\sigma(x,y)$   		&\verb|sigma| &  	Englacial void fraction connected to sheet\\
$\lambda(x,y)$   	&\verb|lcx|,\verb|lcy|,\verb|lcs|,\verb|lcr|  & Sheet width contributing to conduit melting\\
\end{tabular} \\
\hline
& \\[-10pt]
\hline
Parameters [\verb|pd|]&
\begin{tabular}[t]{l l l}
$\rho_w$ 		&\verb|rho_w| &  Density of water [$1000\un{kg}\un{m}^{-3}$]\\ 
$\rho_i$ 		&\verb|rho_i| &  Density of ice [$910\un{kg}\un{m}^{-3}$]\\ 
$g$			&\verb|g| &  Gravitational acceleration [$9.81\un{m}\un{s}^{-2}$]\\
$L$			&\verb|L| &  Latent heat of melting [$3.35\times 10^5\un{J}\un{kg}^{-3}$] \\
$c$ 			&\verb|c| &  Specific heat capacity of water [$4200 \un{J} \un{kg}^{-1}\un{K}^{-1}$]\\
$\beta$ 		&\verb|gamma_cc| &  Melting point pressure gradient [$0 \un{K}\un{Pa}^{-1}$]\\
$G$			&\verb|G| &  Geothermal heat flux [$0.063\un{W}\un{m}^{-2}$]\\
$n$			&\verb|n_Glen| &  Ice flow law exponent [$3$] \\
$A$         		&\verb|A| &  Ice flow law coefficient [$6.8\times 10^{-24} \un{Ps}^{-3} \un{s}^{-1}$]\\
$\tilde{A}$         	&\verb|K_s| &  Modified ice flow law coefficient in channel [$5.04\times 10^{-25} \un{Ps}^{-3} \un{s}^{-1}$]\\
$\hat{A}$         	&\verb|K_c| &  Modified ice flow law coefficient in sheet [$5.04\times 10^{-25} \un{Ps}^{-3} \un{s}^{-1}$]\\
$\alpha_c$        &\verb|alpha_c| &  Conduit flux exponent [$5/4$]\\
$\beta_c$          &\verb|beta_c| &  Conduit pressure gradient exponent [$1/2$]\\
$\alpha_s$ 	&\verb|alpha_s| &  Sheet flux exponent [$3$]\\
$\beta_s$          &\verb|beta_s| &  Sheet pressure gradient exponent [$1$]\\
%$\alpha_e$ 	&\verb|alpha_e| &  Elastic sheet flux exponent [$3$]\\
%$\beta_e$          &\verb|beta_e| &  Elastic sheet pressure gradient exponent [$1$]\\
$K_c$		&\verb|k_c| &  Conduit flux coefficient [$0.1\un{m} \un{s}^{-1} \un{Pa}^{-1/2}$]\\
$K_s$		&\verb|k_s| &  Sheet flux coefficient [$10^{-4} \un{Pa}^{-1}\un{s}^{-1}$] \\
%$K_s$		&\verb|k_s| &  Elastic sheet flux coefficient [$10^{-4} \un{Pa}^{-1}\un{s}^{-1}$] \\
$h_r$		&\verb|h_r| &  Sheet roughness height [$0.1\un{m}$] \\
$\ell_r$		&\verb|l_r| &  Sheet roughness length [$10\un{m}$] \\
$\lambda$	&\verb|l_c| &  Default sheet width contributing to conduit melting [$10\un{m}$] \\
$U_b$		&\verb|u_b| &  Default basal sliding speed [$60\un{m}/\un{y}$] \\
$\sigma$		&\verb|sigma| &  Default englacial void fraction [$0$] \\
$m$			&\verb|melt| &  Default basal melt rate [$0.0059\un{m}/\un{y}$] \\
$A_m$		&\verb|A_m| &  Moulin cross-sectional area [$10\un{m}^2$] \\
$h_{rc}$		&\verb|h_rc| &  Conduit roughness height [$0\un{m}$] \\
$S_{rc}$		&\verb|S_rc| &  Conduit area cutoff [$0\un{m}^2$] \\
$\gamma$ 	&\verb|gamma_e| &  Elastic sheet exponent [$1$]\\
$h_c$		&\verb|c_e_power| &  Elastic sheet depth scale [$0\un{m}$] \\
$h_{\eps 2}$	&\verb|c_e_reg2| & Uplift regularization rate [$0\un{m}\un{Pa}^{-1}$] \\
$N_{\eps 2}$	&\verb|N_reg2| & Regularizing pressure for uplift regularization [$10^3\un{Pa}$] \\
$\Psi_{\eps}$   &\verb|Psi_reg| &  Regularizing potential gradient [$0.1\un{Pa}\un{m}^{-1}$] \\
$N_{\eps}$	&\verb|N_reg| & Regularizing pressure for elastic sheet [$10^3\un{Pa}$] \\
$p_{\eps}$	&\verb|p_a_reg| & Pressure tolerance for boundary adjustment [$9810\un{Pa}$] \\

\end{tabular}  \\
\hline
\end{tabular}
}
\caption{\label{tab1}{\small Variables, prescribed inputs, and parameters with default values. Note that several of the parameters have different names in the code due to historical legacies.}}
\end{table}

%\clearpage

\subsection*{Numerical structure of the model}

An example model run is given in {\verb|nevis_example.m|}.

The computations are all performed using non-dimensional variables; that is, each variable is scaled by an appropriate value.  This non-dimensionalization is performed before calculations begin, and the variables that are subsequently used and saved are all non-dimensional.  In order to plot the dimensional quantities, the quantities must be multiplied by the appropriate scale. 

Everything is stored in Matlab structures:

\hangindent0.5cm\hangafter=0  
\noindent {\verb pd } dimensional parameters \\
 {\verb oo } options \\
 {\verb ps } scales used for non-dimensionalization \\
 {\verb pp } parameters (non-dimensional) \\
 {\verb gg } grid and discretization (non-dimensional) \\
 {\verb aa } prescribed fields and boundary conditions (non-dimensional) \\
 {\verb vv } current solution variables (non-dimensional) \\
 {\verb tt } time series of solution summary (non-dimensional) \\
 
Default parameters and options are assigned and then non-dimensionalized by
\begin{verbatim}
>pd = struct; oo = struct;
>[pd,oo] = nevis_defaults(pd,oo);
>[ps,pp] = nevis_nondimension(pd);
\end{verbatim}
Non-default values can be assigned to \verb|pd| or \verb|oo| before calling \verb|nevis_defaults|.

A grid is set up by 
\begin{verbatim}
>x = linspace(0,10000/ps.x,50); y = linspace(0,10000/ps.x,50);
>gg = nevis_grid(x,y,oo);
>gg = nevis_mask(gg,nout);
>gg = nevis_label(gg,nbdy);
\end{verbatim}
\verb|nevis_grid| populates \verb|gg| with the coordinates of nodes and edges, as well as derivative and mean matrix operators.  \verb|x| and \verb|y| define the $x$ and $y$ coordinates for the rectangular grid (note the scaling by \verb|ps.x| to make the quantities non-dimensional).  \verb|oo| can include optional flags for how to assign the coordinates.

\verb|nevis_mask| adds labels to \verb|gg| to identify which nodes and edges are inside, outside, and on the boundary of the domain, using the node indices \verb|nout| to define the region outside the domain.  

\verb|nevis_label| re-labels the boundary nodes \verb|nbdy| at which the hydraulic potential is to be prescribed, and the adjoining edges as inside, outside or on the boundary of the domain. 

The definition of \verb|nout| and \verb|nbdy| depend on the specific problem and may for instance be defined based on ice thickness, or using a pre-existing mask.

%An alternative (old) way of calling \verb|nevis_grid| is
%\begin{verbatim}
%>gg = nevis_grid(nI,nJ,1,xl,xr,yb,yt,oo);
%\end{verbatim}
%\verb|nevis_grid| populates \verb|gg| with the coordinates of nodes and edges, as well as derivative and mean matrix operators.  \verb|nI| and \verb|nJ| are the number of nodes to place in the $x$ and $y$ directions, and \verb|xl|, \verb|xr|, \verb|yb| and \verb|yt| define the minimum and maximum values of $x$ and $y$.  \verb|oo| can include optional flags for how to assign the coordinates.

Prescribed fields and initial conditions are assigned by
\begin{verbatim}
>b = (0/ps.z)*gg.nx.^0; 
>s = (100/ps.z)*gg.nx.^0;
>[aa,vv] = nevis_initialize(b,s,gg,pp,oo);
\end{verbatim}
\verb|nevis_initialize| populates \verb|aa| with default prescribed fields, and \verb|vv| with default initial conditions for the primary variables (generally these won't be sensible initial conditions, so new ones should be assigned subsequently).  \verb|b| and \verb|s| are matrices or vectors defining the bed and surface elevation, here taken to be uniform at $0\un{m}$ and $100\un{m}$ for illustration.

The model is solved by
\begin{verbatim}
>[tt,vv] = nevis_timesteps(t_span,vv,aa,pp,gg,oo);
\end{verbatim}
\verb|nevis_timesteps| solves the equations over the timespan \verb|t_span|, with initial conditions defined by input \verb|vv|. The output \verb|vv| contains the solution at the final time, and \verb|tt| contains basic summary information about the solution at each timestep.  Depending on the options in \verb|oo|, the solution at intermediate times in \verb|t_span| is also saved to file.

The current solution (defined by contents of \verb|vv|) can be plotted using
\begin{verbatim}
>nevis_plot;
\end{verbatim}
(what is actually plotted depends on options in \verb|oo|; see \verb|nevis_plot.m|), or by typing things like
\begin{verbatim}
>imagesc(ps.x*gg.nx,ps.x*gg.ny,ps.phi*reshape(phi,gg.nI,gg.nJ));
\end{verbatim}
(note how the scalings in \verb|ps| are used to convert the non-dimensional quantities to dimensional ones).

\subsection*{Domain and boundary conditions}

Nodes are labelled as inside the solution domain (indices \verb|gg.ns|) or outside (\verb|gg.nout|).  Similar labels are assigned to the edges and corners.  Those indices on which boundary conditions are prescribed are labelled \verb|gg.nbdy|, \verb|gg.ebdy|, or \verb|gg.fbdy|.  The method of labelling is described below.

By default, the hydraulic potential on indices \verb|gg.nbdy| is set to \verb|phi_a|, and the discharge on all boundary edges \verb|gg.ebdy|, \verb|gg.fbdy| is set to \verb|0|.   If other values are to be assigned, these should be included in \verb|aa|; {\it e.g.} \verb|aa.phi| can be defined as a list of values for the hydraulic potential on the nodes \verb|gg.nbdy|.

By default the exterior boundaries of the grid set up by \verb|nevis_grid| are reflecting and there are no boundary nodes or edges (the exterior boundaries can instead be made periodic using \verb|oo.xperiodic| and \verb|oo.yperiodic|).  Usually, however, a solution domain that is strictly inside the rectangular grid is defined, so that these exterior boundaries are irrelevant.  This is performed using
\begin{verbatim}
>gg = nevis_mask(gg,nout);
\end{verbatim}
\verb|nevis_mask| labels node indices \verb|nout| as outside the domain, the remaining nodes as inside the domain, edges that are connected to at least one inside node as inside the domain, and corners that are surrounded by inside nodes as inside the domain.  It also identifies which nodes are on the boundary of the domain (\verb|gg.n1|), which edges connect inside nodes to outside nodes (\verb|gg.e1|, \verb|gg.f1|), and by default it labels these edges as the boundary edges \verb|gg.ebdy|, \verb|gg.fbdy| on which discharge is prescribed.

To prescribe pressure on certain nodes (usually those at the margin), use
\begin{verbatim}
>gg = nevis_label(gg,nbdy);
\end{verbatim}
which assigns the list of node indices \verb|nbdy| as the boundary nodes \verb|gg.nbdy| and removes the adjoining edges from the labels \verb|gg.ebdy|, \verb|gg.fbdy|.

If the option \verb|oo.adjust_boundaries| is set, the labelling of boundary nodes may be adjusted dynamically during timestepping (see below).  In that case, a list of current boundary nodes is stored in \verb|vv.nbdy|, and the grid labelling is updated at each timestep using \verb|nevis_label|.  This option is incompatible with prescribing non-default boundary values in \verb|aa|.

\subsection*{Initial conditions}

The default initial conditions assigned by \verb|nevis_initialize| have zero sheet and conduit sizes, which will usually not work.  Starting with a uniform sheet and with pressure at some percentage of overburden often seems to work,
\begin{verbatim}
>vv.phi = aa.phi_a + 0.9*(aa.phi_0-aa.phi_a);
>vv.hs = (0.1/ps.h)*gg.nx.^0;
\end{verbatim} 
It is usually a good idea to run the model with a constant input for some time before beginning calculations in earnest, to avoid a significant transient from the initial condition.

\subsection*{Moulins}

Moulins are defined with a list of node indices \verb|pp.ni_m| for the moulins.  In addition, either a function handle \verb|pp.input_function(t)| defining the input to the moulins, or a matrix \verb|pp.sum_m| defining the catchment areas, should be provided.  A list of corresponding moulin labels \verb|pp.num_m| can also be defined.  The function 
\begin{verbatim}
>[pp.ni_m,pp.sum_m] = nevis_moulins(x_m,y_m,gg,oo);
\end{verbatim}
finds the node indices and the catchment areas (based on a Voronoi tesselation) for moulins with coordinates \verb|x_m|,\verb|y_m|.  If \verb|oo.random_moulins| is non zero, that number of moulins will be located at random.


\subsection*{Inputs}

The input \verb|aa.E| is updated at each time \verb|t| using the function
\begin{verbatim}
>aa = nevis_inputs(t,aa,pp,gg,oo);
\end{verbatim}
\verb|nevis_inputs| assigns the input in different ways depending on the options in \verb|oo|:

If no moulins have been defined, the default is to use function handle \verb|pp.meltE(t)| and lapse rate \verb|pp.E_lapse|, modulated by diurnal amplitude \verb|pp.E_amp|, to define a distributed runoff.

If moulins have been defined ({\it i.e.} \verb|pp.ni_m| is non-empty), the default is to calculate the same runoff but concentrate it into the moulins according to catchment areas defined by \verb|pp.sum_m| (distributed input can still be enforced by setting \verb|oo.distributed_input|).

If \verb|oo.runoff_function|, the runoff is instead calculated using the function handle \verb|pp.runoff_function(t)| instead (this function should return runoff on each node).

If \verb|oo.input_function|, the function handle \verb|pp.input_function(t)| is used to define the input to the moulins (this function should return runoff to each moulin). This overrides the options above.

%There is also the possibility of prescribing a boundary input \verb|pp.Q_in(t)| to a set of boundary nodes labelled by \ver|pp.ei_in| (which references entries in \verb|gg.ebdy|).

\subsection*{Timestepping}

Timestepping is performed using
\begin{verbatim}
>[tt,vv] = nevis_timesteps(t_span,vv,aa,pp,gg,oo);
\end{verbatim}
The start and end times are defined by the first and last entries of vector \verb|t_span| and the initial conditions are defined by \verb|vv|.  The initial timestep is defined by \verb|oo.dt|.  The procedure at each time step is first to update the inputs for the current time; then calculate the summary information save in \verb|tt| (see below); possibly write the current solution to file (see below); then attempt a timestep (with a Newton iteration performed by the function \verb|nevis_timestep|; adjusting the timestep if required, and increasing the suggested next timestep if the iteration was very quick); then finally adjusting the boundary nodes for the next timestep (see below). 

The bulk of the work in solving the equations is in the function \verb|nevis_timestep|, which performs a Newton iteration and calls on the function \verb|nevis_backbone| to evaluate the residuals and Jacobian.  This function also calculates the derived variables such as discharge (on the nodes) and effective pressure.  They can be added to the solution structure \verb|vv| by
\begin{verbatim}
>[vv] = nevis_backbone(inf,vv,vv,aa,pp,gg,oo);
\end{verbatim}
If \verb|oo.change_timestep|, the timestep may be adjusted based upon the failure or success of previous iterations (see \verb|nevis_timesteps.m|).  

The structure \verb|tt| stores summary information for every timestep taken.  This includes total inflow \verb|Q_in|, \verb|m| and \verb|E|, total outflow \verb|Q_out|, spatially averaged hydraulic potential \verb|phi| and effective pressure \verb|N|, total cavity sheet volume \verb|hs| and channel volume \verb|S|.  

If the option \verb|oo.save_pts_all|, the hydraulic potential \verb|pts_phi| and cavity sheet depth \verb|pts_hs| are saved at nodes defined by \verb|pp.pts_ni| (this is useful for storing a time series of pressure in moulins, for example).

At each time in the vector \verb|t_span|, the current solution \verb|vv| and the summary structure \verb|tt| are saved to file, with filename defined with the string \verb|oo.fn|.  In addition, if the option \verb|oo.save_timesteps|, the current solution \verb|vv| is saved to a file in a directory with name \verb|oo.fn|, each such file being labelled sequentially.  Thus the number and frequency of entries in \verb|t_span| determines how often the solution is saved.

If \verb|oo.adjust_boundaries|, the discharge into and out of the boundary nodes \verb|gg.nbdy| is evaluated after each timestep.  If there is inflow, this suggests the pressure should not be prescribed there and such nodes are eliminated from \verb|gg.nbdy| for the next timestep.  Conversely, if the pressure at a node that is on margin of the domain (defined by indices \verb|gg.n1m|) is above atmospheric pressure, it suggests that pressure should be prescribed there and such nodes are added to \verb|gg.nbdy| for the next timestep.  Since the list of boundary nodes now changes in time, the current list is stored as \verb|vv.nbdy|.

\subsection*{Plotting}

The current solution (contained in \verb|vv|) is rescaled and plotted using
\begin{verbatim}
>nevis_plot;
\end{verbatim}
By default, this plots the average magnitude of the discharge on each node, expressed as an areal discharge (units $\un{m}^2\un{s}^{-1}$) as if the conduit discharge were spread out over each grid cell. This average is calculated and added to \verb|vv| using 
\begin{verbatim}
>vv = nevis_nodedischarge(vv,aa,pp,gg,oo);
\end{verbatim}
Other things such as the topography can be plotted by setting, {\it e.g.}, \verb|oo.topography| (see \verb|nevis_plot.m|). 

A series of timesteps (say 10 of them) saved into a directory with name \verb|fn| can be animated using
\begin{verbatim}
>nevis_animate(fn,1:10,1,0);
\end{verbatim}
The second input here is the timesteps to plot, the third input is an option to plot the discharge, and the fourth is an option not to save any of the frames.

\end{document}

\begin{figure}
\centering
\includegraphics[width=\textwidth]{}
\caption{}
\end{figure}

\begin{table}
\centering
\begin{tabular}{|c|c|}
\hline
Parameter & Value \\
\hline
$$ & $$\\ 

\hline
\end{tabular}
\caption{\lab{parametervalues}Values of model constants}
\end{table}
